\section{ЗАДАНИЕ 1}

Написать предикат set-equal, который возвращает t, если два
его множества-аргумента содержат одни и те же элементы,
порядок которых не имеет значения.

\begin{lstlisting}
(defun set-equal (list1 list2)
    (let
        (
         (l1 (sort list1 #'<))
         (l2 (sort list2 #'<))
        )
        (equal l1 l2)
    )
)

(set-equal '(1 2 3 4) '(4 3 2 1)) ;;; T
(set-equal '(1 2 3) '(3 2 4)) ;;; Nil
\end{lstlisting}

\section{ЗАДАНИЕ 2}

Напишите необходимые функции, которые обрабатывают
таблицу из точечных пар: (страна. столица), и возвращают по
стране -- столицу, а по столице -- страну.

\begin{lstlisting}
(setq countries '(Russia USA GB Belarus))
(setq cities    '(Moscow Washington London Minsk))

(defun cons_merge (countries cities)
        (mapcar #'(lambda (ctr cty) (cons ctr cty)) countries cities)
)

(setq cons_cc (cons_merge countries cities))

(defun cons_city (cons_cc country)
    (reduce #'(lambda (a b) (or a b))
        (mapcar #'(lambda (el)
                    (and (equal (car el) country) (cdr el))
                ) cons_cc
        )
    )
)

(cons_city cons_cc 'GB) ;;; London
(cons_city cons_cc 'NotExist) ;;; Nil
(cons_city cons_cc 'Russia) ;;; Moscow

(defun cons_country (cons_cc city)
    (reduce #'(lambda (a b) (or a b))
        (mapcar #'(lambda (el)
                    (and (equal (cdr el) city) (car el))
                ) cons_cc
        )
    )
)

(cons_country cons_cc 'London) ;;; GB
(cons_country cons_cc 'NotExist) ;;; Nil
(cons_country cons_cc 'Moscow) ;;; Russia
\end{lstlisting}

\begin{lstlisting}
(setq countries '(Russia USA GB Belarus))
(setq cities '(Moscow Washington London Minsk))

(defun cons_merge_cc (countries cities)
    (cond
        ((null countries)  Nil)
        ((null cities)     Nil)
        (T (cons
            (cons (car countries) (car cities))
            (cons_merge_cc (cdr countries) (cdr cities))
        ))
    )
)

(setq cons_merged_cc (cons_merge_cc countries cities))

(defun cons_find_country (list_cc city)
    (cond
        ((null list_cc) Nil)
        ((equal (cdar list_cc) city) (caar list_cc))
        (T (cons_find_country (cdr list_cc) city))
    )
)

(cons_find_country cons_merged_cc 'Moscow) ;;; Russia
(cons_find_country cons_merged_cc  'Minsk) ;;; Belarus
(cons_find_country cons_merged_cc 'Sidney) ;;; Nil

(defun cons_find_city (list_cc country)
    (cond
        ((null list_cc) Nil)
        ((equal (caar list_cc) country) (cdar list_cc))
        (T (cons_find_city (cdr list_cc) country))
    )
)

(cons_find_city cons_merged_cc    'Russia) ;;; Moscow
(cons_find_city cons_merged_cc   'Belarus) ;;; Minsk
(cons_find_city cons_merged_cc 'Australia) ;;; Nil
\end{lstlisting}

\section{ЗАДАНИЕ 3}

Напишите функцию, которая умножает на заданное число-аргумент
все числа из заданного списка-аргумента, когда

\begin{enumerate}
    \item[а)] все элементы списка -- числа,
    \item[б)] элементы списка -- любые объекты.
\end{enumerate}

\begin{lstlisting}
(defun number_multiplication_car (lst n)
    (mapcar #'(lambda (el) (* el n)) lst)
)

(number_multiplication_car '(1 2 3 4) '5) ;;; (5 10 15 20)

(defun number_multiplication_rec (lst n)
    (cond
        ((null lst) Nil)
        ('T (cons (* n (car lst)) (multiplication_rec (cdr lst) n)))
    )
)

(number_multiplication_rec '(1 2 3 4) '5) ;;; (5 10 15 20)

(defun multiplication_car (lst n)
    (mapcar #'(lambda (el) (if (numberp el) (* el n) el)) lst)
)

(multiplication_car '(1 2 3 4) '5) ;;; (5 10 15 20)
(multiplication_car '(a 2 b 4) '5) ;;; (A 10 B 20)

(defun multiplication_rec (lst n)
    (cond
        ((null lst) Nil)
        ('T
            (let*
                (
                    (el (car lst))
                    (final_el (if (numberp el) (* el n) el))
                    (next_iteration (multiplication_rec (cdr lst) n))
                )
                (cons final_el next_iteration)
            )
        )
    )
)

(multiplication_rec '(1 2 3 4) '5) ;;; (5 10 15 20)
(multiplication_rec '(a 2 b 4) '5) ;;; (A 10 B 20)
\end{lstlisting}

\section{ЗАДАНИЕ 4}

Напишите функцию, которая уменьшает на 10 все числа из списка
аргумента этой функции.

\begin{lstlisting}
(defun minus_ten_car (lst)
    (mapcar #'(lambda (el) (- el 10)) lst)
)

(minus_ten_car '(1 2 3 4 5)) ;;; (-9 -8 -7 -6 -5)

(defun minus_ten_rec (lst)
    (cond
        ((null lst) Nil)
        ('T (cons (- (car lst) 10) (minus_ten_car (cdr lst))))
    )
)

(minus_ten_rec '(1 2 3 4 5)) ;;; (-9 -8 -7 -6 -5)
\end{lstlisting}

\section{ЗАДАНИЕ 5}

Написать функцию, которая возвращает первый аргумент списка-аргумента.
который сам является непустым списком.

\begin{lstlisting}
(defun first_list_map (lst)
    (reduce #'
        (lambda (el1 el2)
            (or
                (and (not (atom el1)) el1)
                (and (not (atom el2)) el2)
            )
        ) lst
    )
)

(first_list_map '(1 (2 3) 4 5 (6 7))) ;;; (2 3)
(first_list_map '(1 2 3 4)) ;;; Nil

(defun first_list_rec (lst)
    (cond
        ((null lst) Nil)
        ((not (atom (car lst))) (car lst))
        ('T (first_list_rec (cdr lst)))
    )
)

(first_list_rec '(1 (2 3) 4 5 (6 7))) ;;; (2 3)
(first_list_rec '(1 2 3 4)) ;;; Nil
\end{lstlisting}

\section{ЗАДАНИЕ 6}

Написать функцию, которая выбирает из заданного списка только те числа,
которые больше 1 и меньше 10.
(Вариант: между двумя заданными границами.)

\begin{lstlisting}
(defun take_between_map (a b lst)
    (reduce #'
        (lambda (lst el)
            (append
                (if (numberp lst)
                    (if (and (<= a lst) (>= b lst))
                        (list lst)
                    )
                    lst
                )
                (if (and (<= a el) (>= b el))
                    (list el)
                )
            )
        ) lst
    )
)

(take_between_map '1 '10 '(-10 -5 0 5 10 15)) ;;; (5 10)
(take_between_map '-10 '5 '(-10 -5 0 5 10 15)) ;;; (-10 -5 0 5)

(defun take_between_rec (a b lst)
    (cond
        ((null lst) Nil)
        ((and (<= a (car lst)) (>= b (car lst))) (cons (car lst) (take_between_rec a b (cdr lst))))
        ('T (take_between_rec a b (cdr lst)))
    )
)

(take_between_rec '1 '10 '(-10 -5 0 5 10 15)) ;;; (5 10)
(take_between_rec '-10 '5 '(-10 -5 0 5 10 15)) ;;; (-10 -5 0 5)
\end{lstlisting}

\section{ЗАДАНИЕ 7}

Написать функцию, вычисляющую декартово произведение двух своих
списков-аргументов. ( Напомним, что А $\times$ В это множество
всевозможных пар (a b), где а принадлежит А, принадлежит В.)

\begin{lstlisting}
(defun decart_map (lst1 lst2)
    (mapcan #'
        (lambda (x)
            (mapcar #'(lambda (y) (cons x y)) lst2)
        ) lst1
    )
)

(decart_map '(1 2 3 4 5) '(a b c d e))
(decart_map '(1 2) '(a b)) ;;; ((1 . A) (1 . B) (2 . A) (2 . B))

(defun decart_one_element (el lst)
    (cond
        ((null lst) Nil)
        ('T (cons (cons el (car lst)) (decart_one_element el (cdr lst))))
    )
)

(defun decart_rec (lst1 lst2)
    (cond
        ((null lst1) Nil)
        ('T (append (decart_one_element (car lst1) lst2) (decart_rec (cdr lst1) lst2)))
    )
)

(decart_rec '(1 2 3 4 5) '(a b c d e))
(decart_rec '(1 2) '(a b)) ;;; ((1 . A) (1 . B) (2 . A) (2 . B))
\end{lstlisting}

\section{ЗАДАНИЕ 8}

Почему так реализовано reduce, в чем причина?

\begin{lstlisting}
(reduce #'+ ()) -> 0
\end{lstlisting}

Поскольку список, к которому применятся функция {\ttfamily reduce},
является пустым, то результатом остается начальное значение,
которое по умолчанию равно 0.

\section{ТЕОРЕТИЧЕСКИЕ ВОПРОСЫ}

\subsubsection{Способы организации повторных вычислений в Lisp}

\begin{itemize}
    \item использование функционалов
    \item использование рекурсии
\end{itemize}

\subsubsection{Различные способы использования функционалов}

{\ttfamily (mapcar/maplist $\#$'func lst)}

{\ttfamily mapcar} -- функция func применяется ко всем элементам списка, начиная
с первого.

{\ttfamily maplist} -- функция func применяется ко всем элементам списка, начиная
с последнего.

{\ttfamily mapcan, mapcon} -- аналогичны mapcar и maplist, используется память
исходных данных, не работают с копиями.

{\ttfamily (reduce $\#$’func lst)}

{\ttfamily reduce} -- функция func применяется каскадным образом
(сначала для первого и второго элемента, потом для результата и следующего и т.д.).

\subsubsection{Что такое рекурсия?
Способы организации рекурсивных функций}

\textbf{Рекурсия} -- это ссылка на определяемый объект во
время его определения.
Т. к. в Lisp используются рекурсивно определенные структуры (списки),
то рекурсия -- это естественный принцип обработки таких структур.

\paragraph{Способы организации рекурсивных функций}

\begin{itemize}
    \item Хвостовая рекурсия. В целях повышения эффективности рекурсивных
        функций рекомендуется формировать результат не на выходе из рекурсии,
        а на входе в рекурсию, все действия выполняя до ухода на следующий шаг
        рекурсии. Это и есть хвостовая рекурсия.
    \item Возможна рекурсия по нескольким параметрам
    \item Дополняемая рекурсия -- при обращении к рекурсивной функции
        используется дополнительная функция не в аргументе вызова , а вне его
    \item Выделяют группу функций множественной рекурсии. На одной
        ветке происходит сразу несколько рекурсивных вызовов. Количество
        условий выхода также может зависеть от задачи.
\end{itemize}

\subsubsection{Способы повышения эффективности реализации рекурсии}

\begin{itemize}
    \item Использование хвостовой рекурсии.
        Если условий выхода несколько, то надо думать о порядке их следования.
    \item Превращение не хвостовой рекурсии в хвостовую.
        Для превращения не хвостовой рекурсии в хвостовую и в целях
        формирования результата (результирующего списка) на входе в
        рекурсию, рекомендуется использовать дополнительные (рабочие)
        параметры. При этом становится необходимым создат фунецию --
        оболочку для реализации очевидного обращения к функции.
\end{itemize}
