\section{ЗАДАНИЕ 1}

Чем принципиально отличаются функции cons, lisp, append?

\begin{lstlisting}
(setf lst1 '(a b))
(setf lst2 '(c d))

(cons lst1 lst2)   ;;; ((A B) C D)
(list lst1 lst2)   ;;; ((A B) (C D))
(append lst1 lst2) ;;; (A B C D)
\end{lstlisting}

Функция cons создает списковую ячейку и кладет в голову первый аргумент, а в хвост --
второй. Функция list создает списковые ячейки для каждого аргумента и соединяет
их в один список. А функция append объединяет два списка в один, состоящий из
элементов двух списков.

\section{ЗАДАНИЕ 2}

Каковы результаты вычисления следующих выражений?

\begin{lstlisting}
(reverse ())         ;;; Nil
(last ())            ;;; Nil
(reverse '(a))       ;;; (A)
(last '(a))          ;;; (A)
(reverse '((a b c))) ;;; ((A B C))
(last '((a b c)))    ;;; ((A B C))
\end{lstlisting}

\section{ЗАДАНИЕ 3}

Написать два варианта функции, которая возвращает последний элемент своего
списка-аргумента.

\begin{lstlisting}
(defun last_red (lst)
    (if (null lst)
        Nil
        (reduce #'(lambda (_ el) el) lst)
    )
)

(last_red ()) ;;; Nil
(last_red '(1 2 3 4 5)) ;;; 5

(defun last_rec (lst)
    (cond
        ((null lst) Nil)
        ((eql (length lst) 1) (car lst))
        (T (last_rec (cdr lst)))
    )
)

(last_rec ()) ;;; Nil
(last_rec '(1 2 3 4 5)) ;;; 5
\end{lstlisting}

\section{ЗАДАНИЕ 4}

Написать два варианта функции, которая возвращает свой список-аргумент без
последнего элемента.

\begin{lstlisting}
(defun list_without_last_map (lst)
    (mapcar #'(lambda (el _) el) lst (cdr lst))
)

(list_without_last_map ()) ;;; Nil
(list_without_last_map '(1 2 3 4 5)) ;;; (1 2 3 4)

(defun list_without_last_rec (lst)
    (cond
        ((null lst) Nil)
        ((eql (length lst) 1) Nil)
        (T (cons (car lst) (list_without_last_rec (cdr lst))))
    )
)

(list_without_last_rec ()) ;;; Nil
(list_without_last_rec '(1 2 3 4 5)) ;;; (1 2 3 4)
\end{lstlisting}

\section{ЗАДАНИЕ 5}

Написать простой вариант игры в кости, в котором бросаются две правильные
кости. Если сумма выпавших очков равна 7 или 11 -- выигрыш, если выпало
(1, 1) или (6, 6) -- игрок получает право снова бросить кости, во всех остальных
случаях ход переходит ко второму игроку, но запоминается сумма выпавших
очков. Если второй игрок не выигрывает абсолютно, то выигрывает тот игрок,
у которого больше очков. Результат игры и значения выпавших костей выводить
на экран с помощью функции print.

\begin{lstlisting}
(defun roll_the_dice ()
    (cons
        (+ (random 6) 1)
        (+ (random 6) 1)
    )
)

(defun turn (n)
    (let
        (
            (roll (roll_the_dice))
        )
        (and
            (if (eql n 1)
                (print "Игрок 1 бросил:")
                (print "Игрок 2 бросил:")
            )
            (print roll)
            (if (or
                    (and (eql (car roll) 1) (eql (cdr roll) 1))
                    (and (eql (car roll) 6) (eql (cdr roll) 6))
                )
                (cons roll (turn n))
                (list roll)
            )
        )
    )
)

(defun sum(res)
    (cond
        ((null res) 0)
        (T
            (let*
                (
                    (el1 (caar res))
                    (el2 (cdar res))
                    (plus (+ el1 el2))
                )
                (if (or (eql plus 7) (eql plus 11))
                    '100000000
                    (+ plus (sum (cdr res)))
                )
            )
        )
    )
)

(defun main ()
    (let
        (
            (res1 (sum (turn 1)))
            (res2 (sum (turn 2)))
        )
        (cond
            ((eql res1 res2) (print "Ничья"))
            ((> res1 res2) (print "Выйграл игрок 1"))
            (T (print "Выйграл игрок 2"))
        )
    )
)

(main)
\end{lstlisting}

\begin{lstlisting}
"Игрок 1 бросил:"
(1 . 4)
"Игрок 2 бросил:"
(1 . 4)
"Ничья"
---

"Игрок 1 бросил:"
(4 . 4)
"Игрок 2 бросил:"
(4 . 6)
"Выйграл игрок 2"
---

"Игрок 1 бросил:"
(3 . 2)
"Игрок 2 бросил:"
(2 . 2)
"Выйграл игрок 1"
---

"Игрок 1 бросил:"
(1 . 1)
"Игрок 1 бросил:"
(4 . 6)
"Игрок 2 бросил:"
(2 . 3)
"Выйграл игрок 1"
"Выйграл игрок 1"
\end{lstlisting}
