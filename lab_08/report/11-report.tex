\section{ЗАДАНИЕ 1}

Написать функцию, которая по своему списку-аргументу lst определяет является ли
он полиндромом (то есть равны ли lst и (reverse lst)).

\begin{lstlisting}
(defun is_palindrom (lst)
    (equal lst (reverse lst))
)

(is_palindrom "abccba") ;;; T
(is_palindrom "abcgba") ;;; Nil
\end{lstlisting}

\section{ЗАДАНИЕ 2}

Напишите функцию swap-first-last, которая переставляет в списке-аргументе
первый и последний элементы.

\begin{lstlisting}
(defun swap-first-last (lst)
    (cond
        ((null lst) Nil)
        ((eql (length lst) 1) lst)
        (T
            (append
                (last lst)
                (mapcar #'(lambda (el _) el) (cdr lst) (cddr lst))
                (list (first lst))
            )
        )
    )
)

(swap-first-last ()) ;;; Nil
(swap-first-last '(1)) ;;; (1)
(swap-first-last '(1 2)) ;;; (2 1)
(swap-first-last '(1 2 3 4 5)) ;;; (5 2 3 4 1)
\end{lstlisting}

\section{ЗАДАНИЕ 3}

Напишите функцию swap-tow-elements, которая переставляет в списке-аргументе
два указанных своими порядковыми номерами элемента в этом списке.

\begin{lstlisting}
(defun swap-two-elements (a b lst)
    (cond
        ((eql a b) lst)
        ((> a b) (swap-two-elements b a lst))
        (T
            (append
                (subseq lst 0 a)
                (list (nth b lst))
                (subseq lst (+ a 1) b)
                (list (nth a lst))
                (subseq lst (+ b 1))
            )
        )
    )
)

(swap-two-elements '2 '4 '(1 2 3 4 5)) ;;; (1 2 5 4 3)
(swap-two-elements '1 '4 '(1 2 3 4 5)) ;;; (1 5 3 4 2)
(swap-two-elements '3 '2 '(1 2 3 4 5)) ;;; (1 2 4 3 5)
\end{lstlisting}

\section{ЗАДАНИЕ 4}

Напишите две функции, swap-to-left и swap-to-right, которые производят круговую
перестановку в сиске-аргументе влево и вправо, соотвественно (на k позиций).

\begin{lstlisting}
(defun swap-to-left-one (lst)
    (append (mapcar #'(lambda (el _) el) (cdr lst) lst) (list (car lst)))
)

(defun swap-to-left (lst k)
    (cond
        ((<= k 0) lst)
        (T (swap-to-left (swap-to-left-one lst) (- k 1)))
    )
)

(swap-to-left '(1 2 3 4 5) 2) ;;; (3 4 5 1 2)
(swap-to-left '(1 2 3 4 5) 5) ;;; (1 2 3 4 5)

(defun swap-to-right-one (lst)
    (append (last lst) (mapcar #'(lambda (el _) el) lst (cdr lst)))
)

(defun swap-to-right (lst k)
    (cond
        ((<= k 0) lst)
        (T (swap-to-right (swap-to-right-one lst) (- k 1)))
    )
)

(swap-to-right '(1 2 3 4 5) 2) ;;; (4 5 1 2 3)
(swap-to-right '(1 2 3 4 5) 5) ;;; (1 2 3 4 5)
\end{lstlisting}

\section{ЗАДАНИЕ 5}

Напишите функцию, которая умножает на заданное число-аргумент
все числа из заданного списка-аргумента, когда

\begin{enumerate}
    \item[а)] все элементы списка -- числа,
    \item[б)] элементы списка -- любые объекты.
\end{enumerate}

\begin{lstlisting}
(defun number_multiplication_car (lst n)
    (mapcar #'(lambda (el) (* el n)) lst)
)

(number_multiplication_car '(1 2 3 4) '5) ;;; (5 10 15 20)

(defun multiplication_car (lst n)
    (mapcar #'(lambda (el) (if (numberp el) (* el n) el)) lst)
)

(multiplication_car '(1 2 3 4) '5) ;;; (5 10 15 20)
(multiplication_car '(a 2 b 4) '5) ;;; (A 10 B 20)
\end{lstlisting}

\section{ЗАДАНИЕ 6}

Напишите функцию, select-between, которая из списка-аргумента, содержащего
только числа, выбирает только те, которые расположены между двумя указанными
границами-аргументами и возращает их в виде списка (упорядоченного по
возрастанию списка чисел).

\begin{lstlisting}
(defun select_between (a b lst)
    (sort (reduce #'
        (lambda (lst el)
            (append
                (if (numberp lst)
                    (if (and (<= a lst) (>= b lst))
                        (list lst)
                    )
                    lst
                )
                (if (and (<= a el) (>= b el))
                    (list el)
                )
            )
        ) lst
    ) #'<)
)

(select_between '1 '10 '(40 5 2 -6 3 1 8)) ;;; (1 2 3 5 8)
(select_between '-10 '5 '(-6 -2 -11 8 0 1 -10 -7 100 20 -4))
;;; (-10 -7 -6 -4 -2 0 1)
\end{lstlisting}
